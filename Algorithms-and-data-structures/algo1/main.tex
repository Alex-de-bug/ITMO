\documentclass{article}
\usepackage[utf8]{inputenc} %кодировка
\usepackage[T2A]{fontenc}
\usepackage[english,russian]{babel} %русификатор 
\usepackage{mathtools} %библиотека матеши
\usepackage[left=1cm,right=1cm,top=2cm,bottom=2cm,bindingoffset=0cm]{geometry} %изменение отступов на листе
\usepackage{amsmath}
\usepackage{graphicx} %библиотека для графики и картинок
\graphicspath{}
\DeclareGraphicsExtensions{.pdf,.png,.jpg}
\usepackage{subcaption}
\usepackage{pgfplots}
\usepackage{listings}

\begin{document}
% НАЧАЛО ТИТУЛЬНОГО ЛИСТА
\begin{center}
    \Large
    Федеральное государственное автономное \\
    образовательное учреждение высшего образования \\ 
    «Научно-образовательная корпорация ИТМО»\\
    \vspace{0.5cm}
    \large
    Факультет программной инженерии и компьютерной техники \\
    Направление подготовки 09.03.04 Программная инженерия \\
    \vspace{1cm}
    \Large
    \textbf{Отчёт по лабораторной работе №3} \\
    По дисциплине «Алгоритмы и структуры данных» (4 семестр)\\
    Первый блок задач\\
    \large
    \vspace{8cm}

    \begin{minipage}{.33\textwidth}
    \end{minipage}
    \hfill
    \begin{minipage}{.4\textwidth}
    
        \textbf{Студент}: \vspace{.1cm} \\
        \ Дениченко Александр P3212 \vspace{.1cm}\\
        \textbf{Преподаватели}:  \vspace{.1cm}\\
        \ Косяков М.С. \\
        \ Тараканов Д.С.
    \end{minipage}
    \vfill
Санкт-Петербург\\ 2024 г.
\end{center}

% КОНЕЦ ТИТУЛЬНОГО ЛИСТА 
\newpage

\section{Задача №A «Агроном-любитель»}
\begin{lstlisting}[frame=single, basicstyle=\ttfamily, breaklines=true, breakatwhitespace=true, postbreak=\mbox{\textcolor{red}{$\hookrightarrow$}\space}]
    #include <iostream>
    #include <vector>
    
    int main()
    {
        int n;
        std::cin >> n;
        std::vector<int> numbers(n);
        for (int i = 0; i < n; ++i)
        {
            std::cin >> numbers[i];
        }
        if (n == 1)
        {
            std::cout << "1 1";
            return 0;
        }
        if (n == 2)
        {
            std::cout << "1 2";
            return 0;
        }
        if (n == 3)
        {
            if (numbers[0] == numbers[1] && numbers[1] == numbers[2] && numbers[2] == numbers[0])
            {
                std::cout << "1 2";
                return 0;
            }
            std::cout << "1 3";
            return 0;
        }
        std::vector<int> ans = {1, 2};
        std::vector<int> max = {1, 2};
        for (int i = 2; i < n; i++)
        {
            if (!(numbers[i - 2] == numbers[i - 1] && numbers[i - 1] == numbers[i] && numbers[i - 2] == numbers[i]))
            {
                ans[1] = i + 1;
            }
            else
            {
                ans[0] = i;
                ans[1] = i + 1;
            }
            if ((max[1] - max[0] + 1) < (ans[1] - ans[0] + 1))
            {
                max = ans;
            }
        }
        std::cout << max[0] << " " << max[1];
        return 0;
    }
    
\end{lstlisting}
Сначала проверяются базовые случаи: если на грядке всего один цветок, если два цветка и если три цветка.
Создаются два вектора, которые будут хранить текущий ответ и максимальный найденный ответ соответственно.
Проходя по всем цветкам на грядке, проверяется условие: если три цветка подряд одного вида не обнаружены, то увеличивается длина текущего участка. 
Если обнаружены три цветка подряд одного вида, то обновляется начало текущего участка.
При каждом шаге обновляется максимальный найденный участок без трех цветков одного вида подряд.
После завершения цикла выводится найденный наиболее длинный участок грядки без трех цветков одного вида подряд.
Данный метод решения пришёл ко мне относительно быстро. Благодаря полному детальному проходу алгоритм может считаться точным.
Данный проход перебирает все вариации ответов к задаче, поэтому является эффективным. Таким образом, алгоритм имеет линейную временную сложность в лучшем и худшем случаях.

\section{Задача №B «Зоопарк Глеба»}
\begin{lstlisting}[frame=single, basicstyle=\ttfamily, breaklines=true, breakatwhitespace=true, postbreak=\mbox{\textcolor{red}{$\hookrightarrow$}\space}]
    #include <iostream>
    #include <stack>
    
    int main()
    {
        std::string input_str;
    
        std::cin >> input_str;
    
        std::stack<int> upper_chars;
        std::stack<int> lower_chars;
        std::stack<char> current_str;
        int answer[input_str.length() / 2];
        int up_chars = 0;
        int lo_chars = 0;
        for (int i = 0; i < input_str.length(); i++)
        {
            if (isupper(input_str[i]))
            {
                up_chars++;
                upper_chars.push(up_chars);
                // std::cout << "add upper - " << input_str[i] << " number - " << up_chars << std::endl;
            }
            else
            {
                lo_chars++;
                lower_chars.push(lo_chars);
                // std::cout << "add lower - " << input_str[i] << " number - " << lo_chars << std::endl;
            }
            if (current_str.empty())
            {
                // std::cout << "empty" << std::endl;
                current_str.push(input_str[i]);
            }
            else
            {
                char last = current_str.top();
                if ((abs(last - input_str[i]) == 32))
                {
                    answer[upper_chars.top() - 1] = lower_chars.top();
                    upper_chars.pop();
                    lower_chars.pop();
                    current_str.pop();
                }
                else
                {
                    current_str.push(input_str[i]);
                }
            }
        }
        if (!current_str.empty())
        {
            std::cout << "Impossible" << std::endl;
            return 0;
        }
        std::cout << "Possible" << std::endl;
        for (int i = 0; i < input_str.length() / 2; i++)
        {
            std::cout << answer[i] << " ";
        }
        std::cout << std::endl;
        return 0;
    }
\end{lstlisting}
Мы проходимся по всем животным и ловушкам и соопоставляем их друг другу, любой элемент можно передвинуть в любое место, поэтому нам важно просто расставить животных и ловушки без пересечений путей.
Идея разделения букв (животных и капканов) на стеки пришла оооочень не сразу, сначала был вообще другой подход, который предусматривал разделение букв в двумерный массив, где будет буква и его id.
Так как я не смог увидеть ошибку в прошлом коде, то решил переписать всё с нуля и в итоге опять увидел ошибку, так как в самом начале нового кода я сделал stack $< char >$ $upper\_chars$;
stack $< char >$ $lower\_chars$; где была опять логика сохранять буквы в их порядке, но далее был сделан вывод, что их нет смысла хранить. Я переделал код под хранение индексов в стеках, но забыл проверить пример с очень длинной строкой и char могут неверно отрабатывать, поэтому было решение сохранять индексы в int.
Алгоритм работает за линейное время O(n).
\section{Задача №С «Конфигурационный файл»}
\begin{lstlisting}[frame=single, basicstyle=\ttfamily, breaklines=true, breakatwhitespace=true, postbreak=\mbox{\textcolor{red}{$\hookrightarrow$}\space}]
    #include <iostream>
    #include <stack>
    #include <unordered_map>
    
    bool is_integer(const std::string &s)
    {
        if (s.empty())
            return false;
        size_t start = (s[0] == '-' || s[0] == '+') ? 1 : 0;
        for (size_t i = start; i < s.length(); ++i)
        {
            if (!std::isdigit(s[i]))
            {
                return false;
            }
        }
        return true;
    }
    
    int main()
    {
        std::unordered_map<std::string, std::stack<int>> map;
        std::stack<std::string> new_var;
        std::string line;
    
        while (std::cin >> line)
        {
            if (line.find('=') != std::string::npos)
            {
                std::string key = line.substr(0, line.find('='));
                std::string value = line.substr(line.find('=') + 1);
                // std::cout << "key: " << key << " value: " << value << std::endl;
                if (is_integer(value))
                {
                    int int_value = std::stoi(value);
                    map[key].push(int_value);
                }
                else
                {
                    if (!map[value].empty())
                    {
                        map[key].push(map[value].top());
                        std::cout << map[value].top() << std::endl;
                    }
                    else
                    {
                        map[key].push(0);
                        std::cout << 0 << std::endl;
                    }
                }
                new_var.push(key);
            }
            else
            {
                if (line == "{")
                {
                    // std::cout << "new area" << std::endl;
                    new_var.push("E0F");
                }
                else if (line == "}")
                {
                    // std::cout << "end new area" << std::endl;
                    while (new_var.top() != "E0F")
                    {
                        // std::cout << "clear " << new_var.top() << " : " << map[new_var.top()].top() << std::endl;
                        map[new_var.top()].pop();
                        new_var.pop();
                    }
                    new_var.pop();
                }
                else
                {
                    // {b=1000} ?? error 1: incorrect cin
                    return 1;
                }
            }
        }
        return 0;
    }
\end{lstlisting}
Данная идея реализации мне пришла сразу, так как я ранее писал парсер на питоне для файлов json. 
Сначала была идея сделать $stack<unordered\_map< string, int > > data;$, которая бы хранила в стеке актуальный регион, причём перед добавлением в стек я должнен был копировать прошлую area, но этот подход требовал слишком много памяти, поэтому я пересмотрел решение.
В итоге я пришел к алгоритму - класть в стек все переменные и ещё отдельно хранить изменения в текущей области. Алгоритм работает в среднем за линейное время O(n).
\section{Задача №D «Профессор Хаос»}
\begin{lstlisting}[frame=single, basicstyle=\ttfamily, breaklines=true, breakatwhitespace=true, postbreak=\mbox{\textcolor{red}{$\hookrightarrow$}\space}]
    #include <iostream>
    #include <stack>
    #include <unordered_map>
    
    int main()
    {
        int start;
        int new_vir;
        int delete_vir;
        int place_cont;
        int days;
    
        std::stack<int> curr;
    
        std::cin >> start >> new_vir >> delete_vir >> place_cont >> days;
    
        int test = start;
    
        for (int i = 0; i < days; i++)
        {
            start = start * new_vir;
            if (start <= delete_vir)
            {
                std::cout << 0;
                return 0;
            }
            else
            {
                start = start - delete_vir;
            }
            if (start >= place_cont)
            {
                start = place_cont;
            }
            if (test == start)
            {
                std::cout << start;
                return 0;
            }
        }
        std::cout << start;
        return 0;
    }
\end{lstlisting}
Алгоритм следующий: мы сразу смотрим все данные и начинаем буквально рассматривать каждый день, именно поэтому алгоритм можно считать точным, так как мы рассматриваем все дни как это и происходило бы в реальной жизни. Было пару трудностей, так как не было проверки на зацикливание, но через несколько тестов понял где ошибка. Сложность алгоритма - линейная.


\end{document}

\documentclass{article}
\usepackage[utf8]{inputenc} %кодировка
\usepackage[T2A]{fontenc}
\usepackage[english,russian]{babel} %русификатор 
\usepackage{mathtools} %библиотека матеши
\usepackage[left=1cm,right=1cm,top=2cm,bottom=2cm,bindingoffset=0cm]{geometry} %изменение отступов на листе
\usepackage{amsmath}
\usepackage{graphicx} %библиотека для графики и картинок
\graphicspath{}
\DeclareGraphicsExtensions{.pdf,.png,.jpg}
\usepackage{subcaption}
\usepackage{pgfplots}
\usepackage{listings}
\usepackage{color}


\newcommand{\descr}[2]{%
    \textbf{Описание: } #1\\
    \textbf{Сложность: } #2
}

\begin{document}
% НАЧАЛО ТИТУЛЬНОГО ЛИСТА
\begin{center}
    \Large
    Федеральное государственное автономное \\
    образовательное учреждение высшего образования \\ 
    «Научно-образовательная корпорация ИТМО»\\
    \vspace{0.5cm}
    \large
    Факультет программной инженерии и компьютерной техники \\
    Направление подготовки 09.03.04 Программная инженерия \\
    \vspace{1cm}
    \Large
    \textbf{Отчёт по лабораторной работе №4} \\
    По дисциплине «Алгоритмы и структуры данных» (4 семестр)\\
    Четвёртый блок задач\\
    \large
    \vspace{8cm}

    \begin{minipage}{.33\textwidth}
    \end{minipage}
    \hfill
    \begin{minipage}{.4\textwidth}
    
        \textbf{Студент}: \vspace{.1cm} \\
        \ Дениченко Александр P3212 \vspace{.1cm}\\
        \textbf{Преподаватели}:  \vspace{.1cm}\\
        \ Косяков М.С. \\
        \ Тараканов Д.С.
    \end{minipage}
    \vfill
Санкт-Петербург\\ 2024 г.
\end{center}

% КОНЕЦ ТИТУЛЬНОГО ЛИСТА 
\newpage

\section{Задача №M «Цивилизация»}
\begin{lstlisting}[language=C++, frame=single, basicstyle=\ttfamily, numbers=left, numberstyle=\tiny]
#include <iostream>
#include <queue>
#include <list>
#include <vector>

using namespace std;   

struct Node {
    int cost;
    int marked;
    int x;
    int y;
    int predecessor;

    Node(int cost = -1, int marked = 0, int x = -1, int y = -1, int pred = -1) 
        : cost(cost), marked(marked), x(x), y(y), predecessor(pred) {}

    bool operator>(const Node& other) const {
        return this->cost > other.cost;
    }
};

using NodeQueue = priority_queue<Node, vector<Node>, greater<Node> >;

int main() {
    int N, M, start_x, start_y, finish_x, finish_y;
    cin >> N >> M >> start_x >> start_y >> finish_x >> finish_y;
    start_x--;
    start_y--;
    finish_x--;
    finish_y--;
    int size = N * M;

    vector<vector<pair<int, int> > > map(N, vector<pair<int, int> >(M));

    string tmp;
    for (int i = 0; i < N; i++) {
        cin >> tmp;
        for (int j = 0; j < M; j++) {
            if (tmp[j] == '.') {
                map[i][j].first = 1;
                map[i][j].second = 0;
            } else if (tmp[j] == 'W') {
                map[i][j].first = 2;
                map[i][j].second = 0;
            } else {
                map[i][j].first = 666;
                map[i][j].second = 0;
            }
        }
    }

    NodeQueue pq_sorted;
    vector<Node> visited(size);

    pq_sorted.push(Node(0, 1, start_x, start_y));
    visited[start_x * M + start_y] = Node(0, 1, start_x, start_y, -1);

    while (true) {
        if(pq_sorted.empty()) break;
        Node curr = pq_sorted.top();
        pq_sorted.pop();

        int curr_index = curr.x * M + curr.y;
        if (curr_index == finish_x * M + finish_y) break;

        
        int new_y = curr.y-1;
        if (new_y >= 0 && new_y < M && map[curr.x][new_y].first != 666 && !map[curr.x][new_y].second) {
            int new_cost = curr.cost + map[curr.x][new_y].first;
            int next_index = curr.x * M + new_y;
            if (visited[next_index].cost == -1 || visited[next_index].cost > new_cost) {
                pq_sorted.push(Node(new_cost, 1, curr.x, new_y, curr_index));
                visited[next_index] = Node(new_cost, 1, curr.x, new_y, curr_index);
                map[curr.x][new_y].second = 1;
            }
        }
        
        new_y = curr.y+1;
        if (new_y >= 0 && new_y < M && map[curr.x][new_y].first != 666 && !map[curr.x][new_y].second) {
            int new_cost = curr.cost + map[curr.x][new_y].first;
            int next_index = curr.x * M + new_y;
            if (visited[next_index].cost == -1 || visited[next_index].cost > new_cost) {
                pq_sorted.push(Node(new_cost, 1, curr.x, new_y, curr_index));
                visited[next_index] = Node(new_cost, 1, curr.x, new_y, curr_index);
                map[curr.x][new_y].second = 1;
            }
        }

        int new_x = curr.x-1;
        if (new_x >= 0 && new_x < N && map[new_x][curr.y].first != 666 && !map[new_x][curr.y].second) {
            int new_cost = curr.cost + map[new_x][curr.y].first;
            int next_index = new_x * M + curr.y;
            if (visited[next_index].cost == -1 || visited[next_index].cost > new_cost) {
                pq_sorted.push(Node(new_cost, 1, new_x, curr.y, curr_index));
                visited[next_index] = Node(new_cost, 1, new_x, curr.y, curr_index);
                map[new_x][curr.y].second = 1;
            }
        }

        new_x = curr.x+1;
        if (new_x >= 0 && new_x < N && map[new_x][curr.y].first != 666 && !map[new_x][curr.y].second) {
            int new_cost = curr.cost + map[new_x][curr.y].first;
            int next_index = new_x * M + curr.y;
            if (visited[next_index].cost == -1 || visited[next_index].cost > new_cost) {
                pq_sorted.push(Node(new_cost, 1, new_x, curr.y, curr_index));
                visited[next_index] = Node(new_cost, 1, new_x, curr.y, curr_index);
                map[new_x][curr.y].second = 1;
            }
        }

    }

    
    if (visited[finish_x * M + finish_y].cost == -1){
        cout << visited[finish_x * M + finish_y].cost << endl;
        return 0;
    } 

    string way;
    int curr = finish_x * M + finish_y;
    while (curr != -1 && visited[curr].predecessor != -1) {
        int pred = visited[curr].predecessor;
        int diff = curr - pred;
        if (diff == 1) way = 'E' + way;   
        else if (diff == -1) way = 'W' + way; 
        else if (diff == M) way = 'S' + way; 
        else if (diff == -M) way = 'N' + way;
        curr = pred;
    }
    
    cout << visited[finish_x * M + finish_y].cost << endl;
    cout << way;
    return 0;
}
\end{lstlisting}
\descr{В начале было сложно понять, что и как эту задачу решать. Но после просмотра нескольких лекций на тему графы, выбор пал на алгоритм Дейкстры. Создаются структуры для хранения узлов сетки, каждый узел инициализируется начальной стоимостью, состоянием и координатами. Далее извлекаем узел с минимальной стоимостью из очереди с приоритетами и рассматривает его соседей клеткой. 
Если переход в соседний узел уменьшает стоимость пути, узел добавляется в очередь с обновленной стоимостью. В конце можно легко восстановить путь от целевой точки.}{
    Сложность алгоритма Дейкстры составляет \( O((V + E) \log V) \), где \( V \) — количество вершин (узлов) в графе, а \( E \) — количество ребер (связей между узлами). 
    В контексте сетки размером \( N \times M \):
    \( V = N \times M \); \( E \) приблизительно равно \( 4 \times V \) (поскольку каждый узел связан с четырьмя соседями).
    Таким образом, сложность алгоритма для сетки составляет \( O((N \times M + 4 \times N \times M) \log (N \times M)) \) или \( O(N \times M \log (N \times M)) \).
}\\

\section{Задача №O «Долой списывание!»}
\begin{lstlisting}[language=C++, frame=single, basicstyle=\ttfamily, numbers=left, numberstyle=\tiny]
#include <iostream>
#include <vector>
#include <stack>
using namespace std;

vector<vector<int> > graph;
int colors[101] = {0};

bool dfs(int start){
    stack<pair<int, int> > s;
    pair<int, int> tmp;
    tmp.first = start;
    tmp.second = 1;
    s.push(tmp);

    while (!s.empty()) {
        pair<int, int> top = s.top(); s.pop();
        int v = top.first;
        int color = top.second;

        if (colors[v] == 0) {
            colors[v] = color;
        }

        for (int next : graph[v]) {
            if (colors[next] == 0) { 
                tmp.first = next;
                if(color == 1){
                    tmp.second = 2;
                }else{
                    tmp.second = 1;
                }
                s.push(tmp);
            } else if (colors[next] == color) {
                return false;
            }
        }
    }
    return true;
}

int main() {
    int N, M, x1, x2;
    cin >> N >> M;
    graph.resize(N+1);  

    for (int i = 0; i < M; i++){
        cin >> x1 >> x2;
        graph[x1].push_back(x2);
        graph[x2].push_back(x1);
    }

    for (int i = 0; i < N; i++){
        if (colors[i] == 0 && !dfs(i)) {
            cout << "NO";
            return 0; 
        }
    }

    cout << "YES"; 
    return 0;
}
    
\end{lstlisting}
\descr{Задача относительно проста, быстро додумался что нужно простро проверить граф на двудольность, вспомнит Полякова с его дискреткой и раскроской графа (все цвета можно взять у человека для раскраски, допустим из уха и из носа будет достаточно для этой задачи). 
Был использован DFS при помощи явного стека (увидел на википедии что-то подобное и решил попробовать использовать суть). Каждый узел представлен парой, где first — это идентификатор узла, а second — цвет для назначения (либо 1, либо 2). 
Проходимся по каждой вершине и если вершина еще не окрашена (colors[v] == 0), то ей присваивается цвет и она помещается в стек.
Пока стек не пуст, программа проверяет каждую вершину v, проверяет её цвет и пытается покрасить смежные вершины в альтернативный цвет. 
Если она встречает смежную вершину, уже окрашенную в тот же цвет, что и v, граф не является двудольным.
}{Операция DFS имеет временную сложность \( O(V + E) \), где \( V \) — количество вершин, а \( E \) — количество ребер.
}

\end{document}
